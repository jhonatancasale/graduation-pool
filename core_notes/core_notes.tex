%%%%%%%%%%%%%%%%%%%%%%%%%%%%%%%%%%%%%%%%%
% Tufte-Style Book (Documentation Template)
% LaTeX Template
% Version 1.0 (5/1/13)
%
% This template has been downloaded from:
% http://www.LaTeXTemplates.com
%
% Original author:
% The Tufte-LaTeX Developers (tufte-latex.googlecode.com)
%
% License:
% Apache License (Version 2.0)
%
% IMPORTANT NOTE:
% In addition to running BibTeX to compile the reference list from the .bib
% file, you will need to run MakeIndex to compile the index at the end of the
% document.
%
%%%%%%%%%%%%%%%%%%%%%%%%%%%%%%%%%%%%%%%%%

%----------------------------------------------------------------------------------------
% PACKAGES AND OTHER DOCUMENT CONFIGURATIONS
%----------------------------------------------------------------------------------------

% Use the tufte-book class which in turn uses the tufte-common class
\documentclass[a4paper, notoc]{tufte-book}

% Comment this line if you don't wish to have colored links
\hypersetup{colorlinks}

% Improves character and word spacing
\usepackage{microtype}

% Inserts dummy text
\usepackage{lipsum}

% Better horizontal rules in tables
\usepackage{booktabs}

% Needed to insert images into the document
\usepackage{graphicx}
% Sets the default location of pictures
\graphicspath{{graphics/}}
% Improves figure scaling
\setkeys{Gin}{width=\linewidth,totalheight=\textheight,keepaspectratio}

% Allows customization of verbatim environments
\usepackage{fancyvrb}
% The font size of all verbatim text can be changed here
\fvset{fontsize=\normalsize}

% New command to create parentheses around text in tables which take up no 
% horizontal space - this improves column spacing
\newcommand{\hangp}[1]{\makebox[0pt][r]{(}#1\makebox[0pt][l]{)}}
% New command to create asterisks in tables which take up no horizontal 
% space - this improves column spacing
\newcommand{\hangstar}{\makebox[0pt][l]{*}}

% Used for printing a trailing space better than using a tilde (~) using 
% the \xspace command
\usepackage{xspace}

\newcommand{\monthyear}{\ifcase\month\or January\or February\or March\or April\or May\or June\or July\or August\or September\or October\or November\or December\fi\space\number\year} % A command to print the current month and year

% This block sets up a command for printing an epigraph with 2 arguments - 
% the quote and the author
\newcommand{\openepigraph}[2]{
\begin{fullwidth}
\sffamily\large
\begin{doublespace}
\noindent\allcaps{#1}\\ % The quote
\noindent\allcaps{#2} % The author
\end{doublespace}
\end{fullwidth}
}

% Command to insert a blank page
\newcommand{\blankpage}{\newpage\hbox{}\thispagestyle{empty}\newpage}

% Used for printing standard units
\usepackage{units}

\newcommand{\hlred}[1]{\textcolor{Maroon}{#1}} % Print text in maroon
\newcommand{\hangleft}[1]{\makebox[0pt][r]{#1}} % Used for printing commands in the index, moves the slash left so the command name aligns with the rest of the text in the index 
\newcommand{\hairsp}{\hspace{1pt}} % Command to print a very short space
\newcommand{\ie}{\textit{i.\hairsp{}e.}\xspace} % Command to print i.e.
\newcommand{\eg}{\textit{e.\hairsp{}g.}\xspace} % Command to print e.g.
\newcommand{\na}{\quad--} % Used in tables for N/A cells
\newcommand{\measure}[3]{#1/#2$\times$\unit[#3]{pc}} % Typesets the font size, leading, and measure in the form of: 10/12x26 pc.
\newcommand{\tuftebs}{\symbol{'134}} % Command to print a backslash in tt type in OT1/T1

\providecommand{\XeLaTeX}{X\lower.5ex\hbox{\kern-0.15em\reflectbox{E}}\kern-0.1em\LaTeX}
\newcommand{\tXeLaTeX}{\XeLaTeX\index{XeLaTeX@\protect\XeLaTeX}} % Command to print the XeLaTeX logo while simultaneously adding the position to the index

\newcommand{\doccmdnoindex}[2][]{\texttt{\tuftebs#2}} % Command to print a command in texttt with a backslash of tt type without inserting the command into the index

\newcommand{\doccmddef}[2][]{\hlred{\texttt{\tuftebs#2}}\label{cmd:#2}\ifthenelse{\isempty{#1}} % Command to define a command in red and add it to the index
{ % If no package is specified, add the command to the index
\index{#2 command@\protect\hangleft{\texttt{\tuftebs}}\texttt{#2}}% Command name
}
{ % If a package is also specified as a second argument, add the command and package to the index
\index{#2 command@\protect\hangleft{\texttt{\tuftebs}}\texttt{#2} (\texttt{#1} package)}% Command name
\index{#1 package@\texttt{#1} package}\index{packages!#1@\texttt{#1}}% Package name
}}

\newcommand{\doccmd}[2][]{% Command to define a command and add it to the index
\texttt{\tuftebs#2}%
\ifthenelse{\isempty{#1}}% If no package is specified, add the command to the index
{%
\index{#2 command@\protect\hangleft{\texttt{\tuftebs}}\texttt{#2}}% Command name
}
{%
\index{#2 command@\protect\hangleft{\texttt{\tuftebs}}\texttt{#2} (\texttt{#1} package)}% Command name
\index{#1 package@\texttt{#1} package}\index{packages!#1@\texttt{#1}}% Package name
}}

% A bunch of new commands to print commands, arguments, environments, classes, etc within the text using the correct formatting
\newcommand{\docopt}[1]{\ensuremath{\langle}\textrm{\textit{#1}}\ensuremath{\rangle}}
\newcommand{\docarg}[1]{\textrm{\textit{#1}}}
\newenvironment{docspec}{\begin{quotation}\ttfamily\parskip0pt\parindent0pt\ignorespaces}{\end{quotation}}
\newcommand{\docenv}[1]{\texttt{#1}\index{#1 environment@\texttt{#1} environment}\index{environments!#1@\texttt{#1}}}
\newcommand{\docenvdef}[1]{\hlred{\texttt{#1}}\label{env:#1}\index{#1 environment@\texttt{#1} environment}\index{environments!#1@\texttt{#1}}}
\newcommand{\docpkg}[1]{\texttt{#1}\index{#1 package@\texttt{#1} package}\index{packages!#1@\texttt{#1}}}
\newcommand{\doccls}[1]{\texttt{#1}}
\newcommand{\docclsopt}[1]{\texttt{#1}\index{#1 class option@\texttt{#1} class option}\index{class options!#1@\texttt{#1}}}
\newcommand{\docclsoptdef}[1]{\hlred{\texttt{#1}}\label{clsopt:#1}\index{#1 class option@\texttt{#1} class option}\index{class options!#1@\texttt{#1}}}
\newcommand{\docmsg}[2]{\bigskip\begin{fullwidth}\noindent\ttfamily#1\end{fullwidth}\medskip\par\noindent#2}
\newcommand{\docfilehook}[2]{\texttt{#1}\index{file hooks!#2}\index{#1@\texttt{#1}}}
\newcommand{\doccounter}[1]{\texttt{#1}\index{#1 counter@\texttt{#1} counter}}

\usepackage{makeidx} % Used to generate the index
\makeindex % Generate the index which is printed at the end of the document

% This block contains a number of shortcuts used throughout the book
\newcommand{\vdqi}{\textit{VDQI}\xspace}
\newcommand{\ei}{\textit{EI}\xspace}
\newcommand{\ve}{\textit{VE}\xspace}
\newcommand{\be}{\textit{BE}\xspace}
\newcommand{\VDQI}{\textit{The Visual Display of Quantitative Information}\xspace}
\newcommand{\EI}{\textit{Envisioning Information}\xspace}
\newcommand{\VE}{\textit{Visual Explanations}\xspace}
\newcommand{\BE}{\textit{Beautiful Evidence}\xspace}
\newcommand{\TL}{Tufte-\LaTeX\xspace}

%\usepackage[brazil]{babel}
\usepackage[utf8]{inputenc}
\usepackage[scale=2.5]{ccicons}

%----------------------------------------------------------------------------------------
% BOOK META-INFORMATION
%----------------------------------------------------------------------------------------

\title{ The result of my\\Journey through an\\under graduation\\in Statistics}

\author[Jhonatan Casale]{Jhonatan Casale} % Author

%\publisher{I don't have a publisher} % Publisher

%----------------------------------------------------------------------------------------

\begin{document}

\frontmatter

%----------------------------------------------------------------------------------------
% EPIGRAPH
%----------------------------------------------------------------------------------------

\thispagestyle{empty}
\openepigraph{Our bodies change our minds, and our minds can change our
behavior, and our behavior can change our outcomes.}{Amy Cuddy}

\vfill

\openepigraph{Eu gosto de quem eu sou enquanto estou fazendo o que estou
fazendo?}{David Cain}

\vfill

\openepigraph{Conheço muitos que não puderam quando deviam, porque não
quiseram quando podiam.}{François Rabelais}

%----------------------------------------------------------------------------------------

\maketitle % Print the title page

%----------------------------------------------------------------------------------------
% COPYRIGHT PAGE
%----------------------------------------------------------------------------------------

\newpage
\begin{fullwidth}
~\vfill
\thispagestyle{empty}
\setlength{\parindent}{0pt}
\setlength{\parskip}{\baselineskip}
\ccbyncsa
\par\smallcaps{typeset with tufte-latex}
%\par\smallcaps{Published by \thanklesspublisher}
%\par\smallcaps{tufte-latex.googlecode.com}

%\par Licensed under the Apache License, Version 2.0 (the ``License''); you may not use this file except in compliance with the License. You may obtain a copy of the License at \url{http://www.apache.org/licenses/LICENSE-2.0}. Unless required by applicable law or agreed to in writing, software distributed under the License is distributed on an \smallcaps{``AS IS'' BASIS, WITHOUT WARRANTIES OR CONDITIONS OF ANY KIND}, either express or implied. See the License for the specific language governing permissions and limitations under the License.\index{license}

%\par\textit{First printing, \monthyear}
\end{fullwidth}

%----------------------------------------------------------------------------------------

\tableofcontents % Print the table of contents

%----------------------------------------------------------------------------------------

\listoffigures % Print a list of figures

%----------------------------------------------------------------------------------------

\listoftables % Print a list of tables

%----------------------------------------------------------------------------------------
% DEDICATION PAGE
%----------------------------------------------------------------------------------------

\cleardoublepage
~\vfill
\begin{doublespace}
\noindent\fontsize{18}{22}\selectfont\itshape
\nohyphenation
Dedicado aos que me ajudaram, mesmo que apenas não me atrapalhando.
\end{doublespace}
\vfill
\vfill

%----------------------------------------------------------------------------------------
% INTRODUCTION
%----------------------------------------------------------------------------------------

\cleardoublepage
\chapter*{Introduction} % The asterisk leaves out this chapter from the table of contents
Introduce myself

%----------------------------------------------------------------------------------------

\mainmatter

%----------------------------------------------------------------------------------------
% IMPORTED CHAPTERS
%----------------------------------------------------------------------------------------

% First semester
\part{1° Período Ideal}

% Second semester
\part{2° Período Ideal}
\chapter[Direcionamento Acadêmico II]{Direcionamento Acadêmico II}
\label{ch:academic_guidance}\index{Direcionamento Acadêmico II}

\chapter[Cálculo II]{Cálculo II}\label{ch:calculus_II}\index{Cálculo II}

\chapter[Laboratório de Introdução à Ciencia da Computação] {Laboratório de 
Introdução à Ciência da Computação} \label{ch:lab_icc_I}
\index{Laboratório de Introdução à Ciência da Computação}

\chapter[Introdução à Ciência da Computação]{Introdução à Ciência da
Computação}\label{ch:icc_I}\index{Introdução à Ciência da Computação}

\chapter[Álgebra Linear]{Álgebra Linear}\label{ch:linear_algebra}
\index{Álgebra Linear}

\chapter[Probabilidade I]{Probabilidade I}\label{ch:probrability_I}
\index{Probabilidade I}


% Third semester
\part{3° Período Ideal}
\chapter[Matrizes Aplicadas à Estatística]{Matrizes Aplicadas à Estatística}
\label{ch:applied_matrices}\index{Matrizes Aplicadas à Estatística}

%%%%%%%%%%%%%%%%%%%%%%%%%%%%%%%%%%%%%%%%%%%%%%%%%%%%%%%%%%%%%%%%%%%%%%%%%%%%%%%

\section[Programa da disciplina]{Programa}\label{sec:matrices_program}
\index{Programa - Matrizes Aplicadas à Estatística}

\subsection{Revisão dos tópicos:}
\begin{enumerate}
  \item {Matrizes, vetores e operações matriciais}
  \item {Determinante e inversa usual}
  \item {Dependência linear e posto de uma matriz}
  \item {Autovalores e autovetores}
\end{enumerate}

\subsection{Novo conteúdo:}
\begin{enumerate}
  \item {Equações lineares e inversas generalizadas}
  \item {Matrizes em blocos, operações e inversas de matrizes em blocos}
  \item {Formas lineares}
  \item {Formas quadráticas e classificações de formas quadráticas}
  \item {Matrizes ortogonais e matrizes idempotentes}
  \item {Matriz de projeção}
  \item {Produto de Kronecker e propriedades}
  \item {Decomposição de matrizes:}
  \begin{itemize}
    \item {Cholesky}
    \item {Decomposição espectral}
    \item {decomposição do valor singular}
  \end{itemize}
  \item {Aspectos computacionais}
  \item {Aplicações em Estatística}
\end{enumerate}

%%%%%%%%%%%%%%%%%%%%%%%%%%%%%%%%%%%%%%%%%%%%%%%%%%%%%%%%%%%%%%%%%%%%%%%%%%%%%%%

\section{Matrizes, vetores e operações matriciais}
\label{sec:matrices_revision}\index{Matrizes, vetores e operações matriciais}
%% TODO

%%%%%%%%%%%%%%%%%%%%%%%%%%%%%%%%%%%%%%%%%%%%%%%%%%%%%%%%%%%%%%%%%%%%%%%%%%%%%%%

\section{Determinante e inversa usual}
\label{sec:matrices_inverse}\index{Determinante e inversa usual}
%% TODO

%%%%%%%%%%%%%%%%%%%%%%%%%%%%%%%%%%%%%%%%%%%%%%%%%%%%%%%%%%%%%%%%%%%%%%%%%%%%%%%

\section{Dependência linear e posto de uma matriz}
\label{sec:matrices_li_ld}\index{Dependência linear e posto de uma matriz}
%% TODO

%%%%%%%%%%%%%%%%%%%%%%%%%%%%%%%%%%%%%%%%%%%%%%%%%%%%%%%%%%%%%%%%%%%%%%%%%%%%%%%

\section{Autovalores e autovetores}
\label{sec:matrices_eigen}\index{Autovalores e autovetores}
%% TODO

%%%%%%%%%%%%%%%%%%%%%%%%%%%%%%%%%%%%%%%%%%%%%%%%%%%%%%%%%%%%%%%%%%%%%%%%%%%%%%%
\section{Rank|Posto}\label{sec:rank}\index{Posto}
\begin{itemize}
  \item {O posto de uma matriz \ma{A} é o maior numero de linhas (ou
    colunas) linearmente independentes de \ma{A}}
  \item {O posto máximo de uma matriz $mxn$ é $min(m, n)$}
  \item {Uma matriz é dita ser de posto completo se seu correspondente posto
    assume o valor do maior posto posível, seja para linhas ou colunas (o que
   atingir o menor valor)}
 \item {Seja \ma(A) uma matriz quadrada de ordem $nxn$, \ma{A} admite
   inversa se e somente se \ma(A) possuir posto $n$ (ou seja, posto completo}
\end{itemize}

%%%%%%%%%%%%%%%%%%%%%%%%%%%%%%%%%%%%%%%%%%%%%%%%%%%%%%%%%%%%%%%%%%%%%%%%%%%%%%%
\section{Regras relevantes}
\begin{itemize}
  \item {\mat{(AB)} = \mat{A}\mat{B}}
  \item {\deter{\ma{A}} = \deter{\mat{A}}}
  \item {\deter{\ma{AB}} = \deter{\ma{A}}\deter{\ma{B}}}
  \item {\deter{\mai{A}} = \invdet{\ma{A}}}
  \item {\ma{AI} = \ma{A} e \ma{xI} = \ma{x}}
\end{itemize}


%%%%%%%%%%%%%%%%%%%%%%%%%%%%%%%%%%%%%%%%%%%%%%%%%%%%%%%%%%%%%%%%%%%%%%%%%%%%%%%


%%%%%%%%%%%%%%%%%%%%%%%%%%%%%%%%%%%%%%%%%%%%%%%%%%%%%%%%%%%%%%%%%%%%%%%%%%%%%%%


%%%%%%%%%%%%%%%%%%%%%%%%%%%%%%%%%%%%%%%%%%%%%%%%%%%%%%%%%%%%%%%%%%%%%%%%%%%%%%%



%%%%%%%%%%%%%%%%%%%%%%%%%%%%%%%%%%%%%%%%%%%%%%%%%%%%%%%%%%%%%%%%%%%%%%%%%%%%%%%





%%%%%%%%%%%%%%%%%%%%%%%%%%%%%%%%%%%%%%%%%%%%%%%%%%%%%%%%%%%%%%%%%%%%%%%%%%%%%%%





%%%%%%%%%%%%%%%%%%%%%%%%%%%%%%%%%%%%%%%%%%%%%%%%%%%%%%%%%%%%%%%%%%%%%%%%%%%%%%%





%%%%%%%%%%%%%%%%%%%%%%%%%%%%%%%%%%%%%%%%%%%%%%%%%%%%%%%%%%%%%%%%%%%%%%%%%%%%%%%





%%%%%%%%%%%%%%%%%%%%%%%%%%%%%%%%%%%%%%%%%%%%%%%%%%%%%%%%%%%%%%%%%%%%%%%%%%%%%%%





%%%%%%%%%%%%%%%%%%%%%%%%%%%%%%%%%%%%%%%%%%%%%%%%%%%%%%%%%%%%%%%%%%%%%%%%%%%%%%%





%%%%%%%%%%%%%%%%%%%%%%%%%%%%%%%%%%%%%%%%%%%%%%%%%%%%%%%%%%%%%%%%%%%%%%%%%%%%%%%





%%%%%%%%%%%%%%%%%%%%%%%%%%%%%%%%%%%%%%%%%%%%%%%%%%%%%%%%%%%%%%%%%%%%%%%%%%%%%%%





%%%%%%%%%%%%%%%%%%%%%%%%%%%%%%%%%%%%%%%%%%%%%%%%%%%%%%%%%%%%%%%%%%%%%%%%%%%%%%%





%%%%%%%%%%%%%%%%%%%%%%%%%%%%%%%%%%%%%%%%%%%%%%%%%%%%%%%%%%%%%%%%%%%%%%%%%%%%%%%





%%%%%%%%%%%%%%%%%%%%%%%%%%%%%%%%%%%%%%%%%%%%%%%%%%%%%%%%%%%%%%%%%%%%%%%%%%%%%%%





%%%%%%%%%%%%%%%%%%%%%%%%%%%%%%%%%%%%%%%%%%%%%%%%%%%%%%%%%%%%%%%%%%%%%%%%%%%%%%%





%%%%%%%%%%%%%%%%%%%%%%%%%%%%%%%%%%%%%%%%%%%%%%%%%%%%%%%%%%%%%%%%%%%%%%%%%%%%%%%





%%%%%%%%%%%%%%%%%%%%%%%%%%%%%%%%%%%%%%%%%%%%%%%%%%%%%%%%%%%%%%%%%%%%%%%%%%%%%%%





%%%%%%%%%%%%%%%%%%%%%%%%%%%%%%%%%%%%%%%%%%%%%%%%%%%%%%%%%%%%%%%%%%%%%%%%%%%%%%%





%%%%%%%%%%%%%%%%%%%%%%%%%%%%%%%%%%%%%%%%%%%%%%%%%%%%%%%%%%%%%%%%%%%%%%%%%%%%%%%





%%%%%%%%%%%%%%%%%%%%%%%%%%%%%%%%%%%%%%%%%%%%%%%%%%%%%%%%%%%%%%%%%%%%%%%%%%%%%%%





%%%%%%%%%%%%%%%%%%%%%%%%%%%%%%%%%%%%%%%%%%%%%%%%%%%%%%%%%%%%%%%%%%%%%%%%%%%%%%%





%%%%%%%%%%%%%%%%%%%%%%%%%%%%%%%%%%%%%%%%%%%%%%%%%%%%%%%%%%%%%%%%%%%%%%%%%%%%%%%





%%%%%%%%%%%%%%%%%%%%%%%%%%%%%%%%%%%%%%%%%%%%%%%%%%%%%%%%%%%%%%%%%%%%%%%%%%%%%%%





%%%%%%%%%%%%%%%%%%%%%%%%%%%%%%%%%%%%%%%%%%%%%%%%%%%%%%%%%%%%%%%%%%%%%%%%%%%%%%%





%%%%%%%%%%%%%%%%%%%%%%%%%%%%%%%%%%%%%%%%%%%%%%%%%%%%%%%%%%%%%%%%%%%%%%%%%%%%%%%





%%%%%%%%%%%%%%%%%%%%%%%%%%%%%%%%%%%%%%%%%%%%%%%%%%%%%%%%%%%%%%%%%%%%%%%%%%%%%%%





%%%%%%%%%%%%%%%%%%%%%%%%%%%%%%%%%%%%%%%%%%%%%%%%%%%%%%%%%%%%%%%%%%%%%%%%%%%%%%%





%%%%%%%%%%%%%%%%%%%%%%%%%%%%%%%%%%%%%%%%%%%%%%%%%%%%%%%%%%%%%%%%%%%%%%%%%%%%%%%





%%%%%%%%%%%%%%%%%%%%%%%%%%%%%%%%%%%%%%%%%%%%%%%%%%%%%%%%%%%%%%%%%%%%%%%%%%%%%%%





%%%%%%%%%%%%%%%%%%%%%%%%%%%%%%%%%%%%%%%%%%%%%%%%%%%%%%%%%%%%%%%%%%%%%%%%%%%%%%%





%%%%%%%%%%%%%%%%%%%%%%%%%%%%%%%%%%%%%%%%%%%%%%%%%%%%%%%%%%%%%%%%%%%%%%%%%%%%%%%





%%%%%%%%%%%%%%%%%%%%%%%%%%%%%%%%%%%%%%%%%%%%%%%%%%%%%%%%%%%%%%%%%%%%%%%%%%%%%%%



\chapter[Inferência Estatística]{Inferência Estatística}
\label{ch:statistical_inference}\index{Inferência Estatística}

%%%%%%%%%%%%%%%%%%%%%%%%%%%%%%%%%%%%%%%%%%%%%%%%%%%%%%%%%%%%%%%%%%%%%%%%%%%%%%%

\section[Programa da disciplina]{Programa}\label{sec:inference_program}
\index{Programa - Inferência Estatística}

\begin{enumerate}
  \item {Convergência em distribuição e em probabilidade (apenas enunciar a definição)}
  \item {Amostras e Distribuições Amostrais}
  \item {Estatísticas, Estimadores e Propriedades dos Estimadores:}
    \begin{itemize}
      \item {erro quadrático médio}
      \item {consistência}
      \item {BAN}
    \end{itemize}
  \item {Estatísticas Suficientes e Conjuntamente Suficientes}
  \item {Critério da fatoração}
  \item {Família Exponencial e Propriedades}
  \item {Desigualdade de Informação}
  \item {Completitude}
  \item {Rao-Blackwell}
  \item {Lehmann-Scheffé}
  \item {Métodos de Estimação e Propriedades dos Estimadores:}
  \begin{itemize}
    \item {Método dos Momentos}
    \item {Máxima Verossimilhança}
  \end{itemize}
  \item {Intervalo de Confiança}
    \begin{itemize}
      \item {Método da Quantidade Pivotal}
      \item {Intervalo para Populações Normais}
    \end{itemize}
  \item {Testes de Hipóteses}
    \begin{itemize}
      \item {Erro do Tipo I e II}
      \item {Função Poder}
      \item {Tamanho do teste}
      \item {Teste da Razão de Verossimilhanças}
      \item {Testes Mais Poderosos}
      \item {Lema de Neyman-Pearson}
      \item {Testes Uniformemente Mais Poderosos}
      \item {Testes de Razão de Verossimilhanças Generalizado}
      \item {Testes de hipóteses para populações normais:} 
      \begin{itemize}
        \item {média}
        \item {variância}
        \item {igualdade de duas e mais médias e variâncias}
        \item {testes qui-quadrados:} 
        \begin{itemize}
          \item {bondade de ajuste}
          \item {independência}
        \end{itemize}
      \end{itemize}
      \item {Testes assintóticos}
    \end{itemize}
\end{enumerate}

%%%%%%%%%%%%%%%%%%%%%%%%%%%%%%%%%%%%%%%%%%%%%%%%%%%%%%%%%%%%%%%%%%%%%%%%%%%%%%%



% Fourth semester
\part{4° Período Ideal}

% Fifth semester
\part{5° Período Ideal}

% Sixth semester
\part{6° Período Ideal}

% Seventh semester
\part{7° Período Ideal}

% Eighth semester
\part{8° Período Ideal}

% Ninth semester
\part{9° Período Ideal}

%----------------------------------------------------------------------------------------

\backmatter

%----------------------------------------------------------------------------------------
% BIBLIOGRAPHY
%----------------------------------------------------------------------------------------

\bibliography{bibliography} % Use the bibliography.bib file for the bibliography
\bibliographystyle{plainnat} % Use the plainnat style of referencing

%----------------------------------------------------------------------------------------

\printindex % Print the index at the very end of the document

\end{document}
