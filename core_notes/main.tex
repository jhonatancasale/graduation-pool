%%%%%%%%%%%%%%%%%%%%%%%%%%%%%%%%%%%%%%%%%
% Tufte-Style Book (Documentation Template)
% LaTeX Template
% Version 1.0 (5/1/13)
%
% This template has been downloaded from:
% http://www.LaTeXTemplates.com
%
% Original author:
% The Tufte-LaTeX Developers (tufte-latex.googlecode.com)
%
% License:
% Apache License (Version 2.0)
%
% IMPORTANT NOTE:
% In addition to running BibTeX to compile the reference list from the .bib
% file, you will need to run MakeIndex to compile the index at the end of the
% document.
%
%%%%%%%%%%%%%%%%%%%%%%%%%%%%%%%%%%%%%%%%%

%----------------------------------------------------------------------------------------
% PACKAGES AND OTHER DOCUMENT CONFIGURATIONS
%----------------------------------------------------------------------------------------

% Use the tufte-book class which in turn uses the tufte-common class
\documentclass[a4paper, notoc]{tufte-book}

% Comment this line if you don't wish to have colored links
\hypersetup{colorlinks}

% Improves character and word spacing
\usepackage{microtype}

% Inserts dummy text
\usepackage{lipsum}

% Better horizontal rules in tables
\usepackage{booktabs}

% Needed to insert images into the document
\usepackage{graphicx}
% Sets the default location of pictures
\graphicspath{{graphics/}}
% Improves figure scaling
\setkeys{Gin}{width=\linewidth,totalheight=\textheight,keepaspectratio}

% Allows customization of verbatim environments
\usepackage{fancyvrb}
% The font size of all verbatim text can be changed here
\fvset{fontsize=\normalsize}

% New command to create parentheses around text in tables which take up no 
% horizontal space - this improves column spacing
\newcommand{\hangp}[1]{\makebox[0pt][r]{(}#1\makebox[0pt][l]{)}}
% New command to create asterisks in tables which take up no horizontal 
% space - this improves column spacing
\newcommand{\hangstar}{\makebox[0pt][l]{*}}

% Used for printing a trailing space better than using a tilde (~) using 
% the \xspace command
\usepackage{xspace}

\newcommand{\monthyear}{\ifcase\month\or January\or February\or March\or April\or May\or June\or July\or August\or September\or October\or November\or December\fi\space\number\year} % A command to print the current month and year

% This block sets up a command for printing an epigraph with 2 arguments - 
% the quote and the author
\newcommand{\openepigraph}[2]{
\begin{fullwidth}
\sffamily\large
\begin{doublespace}
\noindent\allcaps{#1}\\ % The quote
\noindent\allcaps{#2} % The author
\end{doublespace}
\end{fullwidth}
}

% Command to insert a blank page
\newcommand{\blankpage}{\newpage\hbox{}\thispagestyle{empty}\newpage}

% Used for printing standard units
\usepackage{units}

\newcommand{\hlred}[1]{\textcolor{Maroon}{#1}} % Print text in maroon
\newcommand{\hangleft}[1]{\makebox[0pt][r]{#1}} % Used for printing commands in the index, moves the slash left so the command name aligns with the rest of the text in the index 
\newcommand{\hairsp}{\hspace{1pt}} % Command to print a very short space
\newcommand{\ie}{\textit{i.\hairsp{}e.}\xspace} % Command to print i.e.
\newcommand{\eg}{\textit{e.\hairsp{}g.}\xspace} % Command to print e.g.
\newcommand{\na}{\quad--} % Used in tables for N/A cells
\newcommand{\measure}[3]{#1/#2$\times$\unit[#3]{pc}} % Typesets the font size, leading, and measure in the form of: 10/12x26 pc.
\newcommand{\tuftebs}{\symbol{'134}} % Command to print a backslash in tt type in OT1/T1

\providecommand{\XeLaTeX}{X\lower.5ex\hbox{\kern-0.15em\reflectbox{E}}\kern-0.1em\LaTeX}
\newcommand{\tXeLaTeX}{\XeLaTeX\index{XeLaTeX@\protect\XeLaTeX}} % Command to print the XeLaTeX logo while simultaneously adding the position to the index

\newcommand{\doccmdnoindex}[2][]{\texttt{\tuftebs#2}} % Command to print a command in texttt with a backslash of tt type without inserting the command into the index

\newcommand{\doccmddef}[2][]{\hlred{\texttt{\tuftebs#2}}\label{cmd:#2}\ifthenelse{\isempty{#1}} % Command to define a command in red and add it to the index
{ % If no package is specified, add the command to the index
\index{#2 command@\protect\hangleft{\texttt{\tuftebs}}\texttt{#2}}% Command name
}
{ % If a package is also specified as a second argument, add the command and package to the index
\index{#2 command@\protect\hangleft{\texttt{\tuftebs}}\texttt{#2} (\texttt{#1} package)}% Command name
\index{#1 package@\texttt{#1} package}\index{packages!#1@\texttt{#1}}% Package name
}}

\newcommand{\doccmd}[2][]{% Command to define a command and add it to the index
\texttt{\tuftebs#2}%
\ifthenelse{\isempty{#1}}% If no package is specified, add the command to the index
{%
\index{#2 command@\protect\hangleft{\texttt{\tuftebs}}\texttt{#2}}% Command name
}
{%
\index{#2 command@\protect\hangleft{\texttt{\tuftebs}}\texttt{#2} (\texttt{#1} package)}% Command name
\index{#1 package@\texttt{#1} package}\index{packages!#1@\texttt{#1}}% Package name
}}

% A bunch of new commands to print commands, arguments, environments, classes, etc within the text using the correct formatting
\newcommand{\docopt}[1]{\ensuremath{\langle}\textrm{\textit{#1}}\ensuremath{\rangle}}
\newcommand{\docarg}[1]{\textrm{\textit{#1}}}
\newenvironment{docspec}{\begin{quotation}\ttfamily\parskip0pt\parindent0pt\ignorespaces}{\end{quotation}}
\newcommand{\docenv}[1]{\texttt{#1}\index{#1 environment@\texttt{#1} environment}\index{environments!#1@\texttt{#1}}}
\newcommand{\docenvdef}[1]{\hlred{\texttt{#1}}\label{env:#1}\index{#1 environment@\texttt{#1} environment}\index{environments!#1@\texttt{#1}}}
\newcommand{\docpkg}[1]{\texttt{#1}\index{#1 package@\texttt{#1} package}\index{packages!#1@\texttt{#1}}}
\newcommand{\doccls}[1]{\texttt{#1}}
\newcommand{\docclsopt}[1]{\texttt{#1}\index{#1 class option@\texttt{#1} class option}\index{class options!#1@\texttt{#1}}}
\newcommand{\docclsoptdef}[1]{\hlred{\texttt{#1}}\label{clsopt:#1}\index{#1 class option@\texttt{#1} class option}\index{class options!#1@\texttt{#1}}}
\newcommand{\docmsg}[2]{\bigskip\begin{fullwidth}\noindent\ttfamily#1\end{fullwidth}\medskip\par\noindent#2}
\newcommand{\docfilehook}[2]{\texttt{#1}\index{file hooks!#2}\index{#1@\texttt{#1}}}
\newcommand{\doccounter}[1]{\texttt{#1}\index{#1 counter@\texttt{#1} counter}}

\usepackage{makeidx} % Used to generate the index
\makeindex % Generate the index which is printed at the end of the document

% This block contains a number of shortcuts used throughout the book
\newcommand{\vdqi}{\textit{VDQI}\xspace}
\newcommand{\ei}{\textit{EI}\xspace}
\newcommand{\ve}{\textit{VE}\xspace}
\newcommand{\be}{\textit{BE}\xspace}
\newcommand{\VDQI}{\textit{The Visual Display of Quantitative Information}\xspace}
\newcommand{\EI}{\textit{Envisioning Information}\xspace}
\newcommand{\VE}{\textit{Visual Explanations}\xspace}
\newcommand{\BE}{\textit{Beautiful Evidence}\xspace}
\newcommand{\TL}{Tufte-\LaTeX\xspace}

\usepackage[brazil]{babel}
\usepackage[utf8]{inputenc}
\usepackage[scale=2.5]{ccicons}

%----------------------------------------------------------------------------------------
% BOOK META-INFORMATION
%----------------------------------------------------------------------------------------

\title{Notas de estudo e \\ acompanhamento} % Title of the book

\author[Jhonatan Casale]{Jhonatan Casale} % Author

%\publisher{I don't have a publisher} % Publisher

%----------------------------------------------------------------------------------------

\begin{document}

\frontmatter

%----------------------------------------------------------------------------------------
% EPIGRAPH
%----------------------------------------------------------------------------------------

\thispagestyle{empty}
\openepigraph{The public is more familiar with bad design than good design. It
is, in effect, conditioned to prefer bad design, because that is what it lives
with. The new becomes threatening, the old reassuring.}{Paul Rand, {\itshape
Design, Form, and Chaos}}

\vfill

\openepigraph{A designer knows that he has achieved perfection not when there
is nothing left to add, but when there is nothing left to take away.}{Antoine
de Saint-Exup\'{e}ry} 

\vfill

\openepigraph{\ldots the designer of a new system must not only be the
implementor and the first large-scale user; the designer should also write the
first user manual\ldots If I had not participated fully in all these
activities, literally hundreds of improvements would never have been made,
because I would never have thought of them or perceived why they were
important.}{Donald E. Knuth}

%----------------------------------------------------------------------------------------

\maketitle % Print the title page

%----------------------------------------------------------------------------------------
% COPYRIGHT PAGE
%----------------------------------------------------------------------------------------

\newpage
\begin{fullwidth}
~\vfill
\thispagestyle{empty}
\setlength{\parindent}{0pt}
\setlength{\parskip}{\baselineskip}
\ccbyncsa
\par\smallcaps{typeset with tufte-latex}
%\par\smallcaps{Published by \thanklesspublisher}
%\par\smallcaps{tufte-latex.googlecode.com}

%\par Licensed under the Apache License, Version 2.0 (the ``License''); you may not use this file except in compliance with the License. You may obtain a copy of the License at \url{http://www.apache.org/licenses/LICENSE-2.0}. Unless required by applicable law or agreed to in writing, software distributed under the License is distributed on an \smallcaps{``AS IS'' BASIS, WITHOUT WARRANTIES OR CONDITIONS OF ANY KIND}, either express or implied. See the License for the specific language governing permissions and limitations under the License.\index{license}

%\par\textit{First printing, \monthyear}
\end{fullwidth}

%----------------------------------------------------------------------------------------

\tableofcontents % Print the table of contents

%----------------------------------------------------------------------------------------

\listoffigures % Print a list of figures

%----------------------------------------------------------------------------------------

\listoftables % Print a list of tables

%----------------------------------------------------------------------------------------
% DEDICATION PAGE
%----------------------------------------------------------------------------------------

\cleardoublepage
~\vfill
\begin{doublespace}
\noindent\fontsize{18}{22}\selectfont\itshape
\nohyphenation
Dedicated to those who appreciate \LaTeX{} and the work of \mbox{Edward R.~Tufte} and \mbox{Donald E.~Knuth}.
\end{doublespace}
\vfill
\vfill

%----------------------------------------------------------------------------------------
% INTRODUCTION
%----------------------------------------------------------------------------------------

\cleardoublepage
\chapter*{Introduction} % The asterisk leaves out this chapter from the table of contents
Introduce myself

%----------------------------------------------------------------------------------------

\mainmatter

%----------------------------------------------------------------------------------------
% IMPORTED CHAPTERS
%----------------------------------------------------------------------------------------

% Second semester
\chapter[Direcionamento Acadêmico I]{Direcionamento Acadêmico I}
\label{ch:academic_guidance}\index{Direcionamento Acadêmico I}

\chapter[Cálculo I]{Cálculo I}\label{ch:calculus_I}\index{Cálculo I}

\section{Lembretes}
\subsection{Regras de Derivação}\index{Single Variable Differentiation Rules}
\newthought{Gerais}
\begin{enumerate}
  \item {\ddx{(c) = 0}}
  \item {\ddx{[cf(x)] = cf'(x)}}
  \item {\ddx{[f(x) \pm g(x)] = f'(x) \pm g'(x)}}
  \item {\ddx{[f(x)g(x)] = f'(x)g(x) + f(x)g'(x)}}
  \item {\ddx{\left[\frac{f(x)}{g(x)}\right] = \frac{f'(x)g(x) - f(x)g'(x)}{[g(x)]^2}}}
  \item {\ddx{f(g(x)) = f'(g(x))g'(x)}}
  \item {\ddx{(x^n) = nx^{n-1}}}
\end{enumerate}

\newthought{Exponencial e Logarítmica}
\begin{enumerate}
  \item {\ddx{(e^x) = e^x}}
  \item {\ddx{(b^x) = b^x \ln b}}
  \item {\ddx{\ln |x| = \frac{1}{x}}}
  \item {\ddx{(\log_b x) = \frac{1}{x \ln b}}}
\end{enumerate}

\newthought{Trigonométricas}
\begin{enumerate}
  \item {\ddx{(\sin x) = \cos x}}
  \item {\ddx{(\cos x) = -\sin x}}
  \item {\ddx{(\tan x) = \sec^2 x}}
  \item {\ddx{(\csc x) = -\csc x \cot x}}
  \item {\ddx{(\sec x) = \sec x \tan x}}
  \item {\ddx{(\cot x) = \csc^2 x}}
\end{enumerate}

\newthought{Funções Especiais: Exponencial e Logarítmica}
\begin{itemize}
  \item {$\log_b x = y \iff b^y = x$}
  \item {$\ln x = \log_e x \textrm{, onde } \ln e = 1$}
  \item {$\ln x = y \iff e^y = x$}
\end{itemize}

\newthought{Equações} de Cancelamento
\begin{itemize}
  \item {$\log_b(b^x) = x$}
  \item {$b^{\log_b x} = x$}
  \item {$\ln(e^x) = x$}
  \item {$e^{\ln x} = x$}
\end{itemize}
\newthought{Equações} de Cancelamento
\begin{enumerate}
  \item {$\log_b (xy) = \log_b x + \log_b y$}
  \item {$\log \left( \frac{x}{y}\right) = \log_b x - \log_b y$}
  \item {$\log_b(x^r) = r \log_b x$}
\end{enumerate}

\chapter[Introdução à Ciência da Computação]{Introdução à Ciência da
Computação}\label{ch:icc_I}\index{Introdução à Ciência da Computação}

\chapter[Laboratório de Introdução à Ciencia da Computação] {Laboratório de 
Introdução à Ciência da Computação} \label{ch:lab_icc_I}
\index{Laboratório de Introdução à Ciência da Computação}

\chapter[Álgebra Linear]{Álgebra Linear}\label{ch:linear_algebra}
\index{Álgebra Linear}

\chapter[Probabilidade I]{Probabilidade I}\label{ch:probrability_I}
\index{Probabilidade I}


%----------------------------------------------------------------------------------------

\backmatter

%----------------------------------------------------------------------------------------
% BIBLIOGRAPHY
%----------------------------------------------------------------------------------------

\bibliography{bibliography} % Use the bibliography.bib file for the bibliography
\bibliographystyle{plainnat} % Use the plainnat style of referencing

%----------------------------------------------------------------------------------------

\printindex % Print the index at the very end of the document

\end{document}
